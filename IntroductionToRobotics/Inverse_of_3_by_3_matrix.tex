\documentclass{article}

% Language setting
\usepackage[english]{babel}

% Set page size and margins
\usepackage[a4paper,top=2cm,bottom=2cm,left=3cm,right=3cm,marginparwidth=1.75cm]{geometry}

% Useful packages
\usepackage{amsmath}

\title{Inverse of 3 by 3  Matrix}
\author{Xuhua Huang}

\begin{document}
\maketitle

\section{Theory}
Finding the inverse of a \(3 \times 3\) matrix \(A\) involves several steps. Given a \(3 \times 3\) matrix:

\[
A = \begin{bmatrix}
a & b & c \\
d & e & f \\
g & h & i
\end{bmatrix}
\]

The inverse \(A^{-1}\) can be found using the following formula:

\[
A^{-1} = \frac{1}{\det(A)} \text{adj}(A)
\]

Where:
\begin{itemize}
  \item {\(\det(A)\) is the determinant of \(A\)}
  \item {\(\text{adj}(A)\) is the adjugate (or adjoint) of \(A\)}
\end{itemize}

\subsection{Calculate the Determinant}

   The determinant of \(A\) is given by:
   \[
   \det(A) = a(ei - fh) - b(di - fg) + c(dh - eg)
   \]

\subsection{Find the Matrix of Minors}

   The minor of an element \(a_{ij}\) is the determinant of the \(2 \times 2\) matrix that remains after removing the \(i\)-th row and \(j\)-th column from \(A\).

   \[
   \text{Minor}(A) = \begin{bmatrix}
   ei - fh & di - fg & dh - eg \\
   bi - ch & ai - cg & ah - bg \\
   bf - ce & af - cd & ae - bd
   \end{bmatrix}
   \]

\subsection{Form the Matrix of Cofactors}

   The cofactor matrix is obtained by applying a checkerboard pattern of signs (alternating + and -) to the matrix of minors.

   \[
   \text{Cofactor}(A) = \begin{bmatrix}
   + & - & + \\
   - & + & - \\
   + & - & +
   \end{bmatrix} \circ \text{Minor}(A)
   \]

   Applying this pattern to the minors matrix:

   \[
   \text{Cofactor}(A) = \begin{bmatrix}
   ei - fh & -(di - fg) & dh - eg \\
   -(bi - ch) & ai - cg & -(ah - bg) \\
   bf - ce & -(af - cd) & ae - bd
   \end{bmatrix}
   \]

\subsection{Transpose the Cofactor Matrix}

   The adjugate (adjoint) of \(A\) is the transpose of the cofactor matrix.

   \[
   \text{adj}(A) = \begin{bmatrix}
   ei - fh & -(bi - ch) & bf - ce \\
   -(di - fg) & ai - cg & -(af - cd) \\
   dh - eg & -(ah - bg) & ae - bd
   \end{bmatrix}
   \]

\subsection{Calculate the Inverse}

   Finally, the inverse of \(A\) is given by:

   \[
   A^{-1} = \frac{1}{\det(A)} \text{adj}(A)
   \]

\section{Calculation Example}
To find the inverse of a \(3 \times 3\) matrix \(A\), we need to follow the steps mentioned above. Here, we'll explicitly go through each step with the given matrix \(A\):

\[
A = \begin{bmatrix}
3 & 2 & -3 \\
3 & 4 & -2 \\
-3 & 2 & 6
\end{bmatrix}
\]

\subsection{Calculate the Determinant}
The determinant of \(A\) is calculated as follows:

\[
\det(A) = a(ei - fh) - b(di - fg) + c(dh - eg)
\]

For our matrix \(A\):

\[
\det(A) = 3 \left(4 \cdot 6 - (-2) \cdot 2\right) - 2 \left(3 \cdot 6 - (-2) \cdot (-3)\right) + (-3) \left(3 \cdot 2 - 4 \cdot (-3)\right)
\]

Calculating the individual terms:

\[
3 \left(24 + 4\right) - 2 \left(18 - 6\right) + (-3) \left(6 + 12\right)
\]

\[
3 \cdot 28 - 2 \cdot 12 - 3 \cdot 18
\]

\[
84 - 24 - 54 = 6
\]

So, the determinant \(\det(A) = 6\).

\subsection{Find the Matrix of Minors}
We calculate the minor for each element in \(A\):

- Minor of \(a_{11}\) (element 3):

\[
\begin{vmatrix}
4 & -2 \\
2 & 6
\end{vmatrix} = 4 \cdot 6 - (-2) \cdot 2 = 24 + 4 = 28
\]

- Minor of \(a_{12}\) (element 2):

\[
\begin{vmatrix}
3 & -2 \\
-3 & 6
\end{vmatrix} = 3 \cdot 6 - (-2) \cdot (-3) = 18 - 6 = 12
\]

- Minor of \(a_{13}\) (element -3):

\[
\begin{vmatrix}
3 & 4 \\
-3 & 2
\end{vmatrix} = 3 \cdot 2 - 4 \cdot (-3) = 6 + 12 = 18
\]

- Minor of \(a_{21}\) (element 3):

\[
\begin{vmatrix}
2 & -3 \\
2 & 6
\end{vmatrix} = 2 \cdot 6 - (-3) \cdot 2 = 12 + 6 = 18
\]

- Minor of \(a_{22}\) (element 4):

\[
\begin{vmatrix}
3 & -3 \\
-3 & 6
\end{vmatrix} = 3 \cdot 6 - (-3) \cdot (-3) = 18 - 9 = 9
\]

- Minor of \(a_{23}\) (element -2):

\[
\begin{vmatrix}
3 & 2 \\
-3 & 2
\end{vmatrix} = 3 \cdot 2 - 2 \cdot (-3) = 6 + 6 = 12
\]

- Minor of \(a_{31}\) (element -3):

\[
\begin{vmatrix}
2 & -3 \\
4 & -2
\end{vmatrix} = 2 \cdot (-2) - (-3) \cdot 4 = -4 + 12 = 8
\]

- Minor of \(a_{32}\) (element 2):

\[
\begin{vmatrix}
3 & -3 \\
3 & -2
\end{vmatrix} = 3 \cdot (-2) - (-3) \cdot 3 = -6 + 9 = 3
\]

- Minor of \(a_{33}\) (element 6):

\[
\begin{vmatrix}
3 & 2 \\
3 & 4
\end{vmatrix} = 3 \cdot 4 - 2 \cdot 3 = 12 - 6 = 6
\]

\subsection{Form the Matrix of Cofactors}
The cofactor matrix is obtained by applying a checkerboard pattern of signs (alternating + and -) to the matrix of minors:

\[
\text{Cofactor}(A) = \begin{bmatrix}
+28 & -12 & +18 \\
-18 & +9 & -12 \\
+8 & -3 & +6
\end{bmatrix}
\]

\subsection{Transpose the Cofactor Matrix}
The adjugate (adjoint) of \(A\) is the transpose of the cofactor matrix:

\[
\text{adj}(A) = \begin{bmatrix}
28 & -18 & 8 \\
-12 & 9 & -3 \\
18 & -12 & 6
\end{bmatrix}
\]

\subsection{Calculate the Inverse}
The inverse of \(A\) is given by:

\[
A^{-1} = \frac{1}{\det(A)} \text{adj}(A) = \frac{1}{6} \begin{bmatrix}
28 & -18 & 8 \\
-12 & 9 & -3 \\
18 & -12 & 6
\end{bmatrix}
\]

Thus, the inverse of \(A\) is:

\[
A^{-1} = \begin{bmatrix}
\frac{28}{6} & \frac{-18}{6} & \frac{8}{6} \\
\frac{-12}{6} & \frac{9}{6} & \frac{-3}{6} \\
\frac{18}{6} & \frac{-12}{6} & \frac{6}{6}
\end{bmatrix}
= \begin{bmatrix}
\frac{14}{3} & -3 & \frac{4}{3} \\
-2 & \frac{3}{2} & -\frac{1}{2} \\
3 & -2 & 1
\end{bmatrix}
\]

So the inverse of matrix \(A\) is:

\[
A^{-1} = \begin{bmatrix}
\frac{14}{3} & -3 & \frac{4}{3} \\
-2 & \frac{3}{2} & -\frac{1}{2} \\
3 & -2 & 1
\end{bmatrix}
\]

\end{document}
